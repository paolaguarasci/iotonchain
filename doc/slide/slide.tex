\documentclass{beamer}
\usepackage[utf8]{inputenc}

\usetheme{Madrid}
\usecolortheme{default}
\usenavigationsymbolstemplate{}
\usepackage[italian]{babel}

%------------------------------------------------------------
%This block of code defines the information to appear in the
%Title page
\title[Iot@Chain] %optional
{Iot@Chain}

% \subtitle{sottotitolo}

\author[Paola Guarasic] % (optional)
{P.~Guarasci\inst{1}}

\institute[Unical] % (optional)
{
  \inst{1}%
  Dipartimento di Matematica e Informatica\\
  Università della Calabria
}

\date[Laurea Magistrale 2024] % (optional)
{Laurea Magistrale, Settembre 2024}

%  \logo{\includegraphics[height=1cm]{overleaf-logo}}


%End of title page configuration block
%------------------------------------------------------------



%------------------------------------------------------------
%The next block of commands puts the table of contents at the
%beginning of each section and highlights the current section:

% \AtBeginSection[]
% {
%   \begin{frame}
%     \frametitle{Table of Contents}
%     \tableofcontents[currentsection]
%   \end{frame}
% }
%------------------------------------------------------------


% - titolo (ci vanno le indicazioni come il frontespizio della tesi)
% - contesto e motivazioni
%   - L’argomento della tua tesi
%   – Quali sono le motivazioni per il problema che è stato
%   risolto/argomento che è stato affrontato
%   – Eventualmente Indicare esplicitamente il problema
%   che è stato risolto
% - contributo
%   – I risultati della tesi in termini di
%   • Sistemi software / Software Systems
% - conclusioni
%   - Hai studiato il problema X
%   - Che lo hai risolto così
%   - Ottenendo questo (meraviglioso) risultato
% - Thanks for your attention!!




\begin{document}
% \frame{\titlepage}




{
\setbeamertemplate{footline}{}
\begin{frame}[noframenumbering]
  \begin{center}
    \textbf{\LARGE Universit\`a della Calabria}\\
    \textbf{Dipartimento di Matematica e Informatica}\\
    \vskip 20pt
      { \huge \bfseries Iot@Chain\\[0.4cm]ASP - Blockchain - IoT}\\[0.2cm]
    \vskip 25pt

    \begin{tabular}{p{7cm}p{3cm}}
      Relatore:           & Candidata:       \\
      Prof.~Mario Alviano & Paola Guarasci   \\
                          & Matricola 231847 \\
    \end{tabular}

    \vskip 40pt
    \vskip 5pt
    Anno Accademico 2023/2024
    \vfill
  \end{center}
\end{frame}
}


%---------------------------------------------------------
\begin{frame}
  \frametitle{Contesto e motivazioni}
  %   - L’argomento della tua tesi
  %   – Quali sono le motivazioni per il problema che è stato
  %   risolto/argomento che è stato affrontato
  %   – Eventualmente Indicare esplicitamente il problema
  %   che è stato risolto
\end{frame}
%---------------------------------------------------------

%---------------------------------------------------------
\begin{frame}
  \frametitle{Contributo}
  %   – I risultati della tesi in termini di
  %   • Sistemi software / Software Systems
\end{frame}
%---------------------------------------------------------

%---------------------------------------------------------
\begin{frame}
  \frametitle{Conclusioni}
  %   - Hai studiato il problema X
  %   - Che lo hai risolto così
  %   - Ottenendo questo (meraviglioso) risultato
\end{frame}
%---------------------------------------------------------


%---------------------------------------------------------
{
\setbeamertemplate{footline}{}
\begin{frame}[noframenumbering]
  \begin{center}
    \vfill{\hfill}

    \centering
    \LARGE Grazie per l'attenzione!

    \vfill{\hfill}
  \end{center}
\end{frame}
}
%---------------------------------------------------------

\end{document}
